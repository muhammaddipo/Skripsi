%versi 2 (8-10-2016) 
\chapter{Pendahuluan}
\label{chap:intro}
   
\section{Latar Belakang}
\label{sec:label}

Salah satu tahapan dari pengembangan perangkat lunak adalah pengujian, pengujian merupakan tahapan yang penting pada pengembangan suatu perangkat lunak, pengujian perangkat lunak adalah  proses menjalankan/menguji perangkat lunak atau aplikasi dengan maksud untuk menemukan \textit{bug} dari perangkat lunak yang sedang dikembangkan. Pengujian merupakan tahapan yang memakan \textit{resource} yang berat, baik itu waktu maupun tenaga kerja, karena tidak bisa sembarang \textit{developer} yang bisa melakukan pengujian, pengujian perangkat lunak dilakukan oleh \textit{developer} ahli yang disebut \textit{Software Quality Assurance}(SQA). SQA ialah orang yang akan melakukan pengujian perangkat lunak. SQA yang menjamin kualitas dari sebuah perangkat lunak, terbebas nya perangkat lunak dari segala \textit{bug} dan \textit{error} sebelum perangkat lunak tersebut digunakan oleh pengguna.	

Ada beberapa teknik untuk melakukan pengujian pada perkembangan perangkat lunak, salah satunya yaitu \textit{Test-Driven Development}(TDD). \textit{Test-driven development} (TDD) adalah praktik pemrograman yang menginstruksikan pengembang perangkat lunak untuk menulis kode baru hanya jika tes otomatis gagal, dan menghilangkan duplikasi, tujuan TDD adalah kode bersih yang berfungsi.[2] TDD  membantu untuk mencapai beberapa karakteristik yang diinginkan dalam perangkat lunak, seperti \textit{decoupling} dan pemisahan masalah, praktik lain harus melengkapi penggunaannya, terutama untuk arsitektur dan koherensi desain.

Pada skripsi ini teknik yang akan digunakan untuk melakukan pengujian pada perangkat lunak yaitu \textit{Behaviour-Driven Development}(BDD). BDD adalah pendekatan pengembangan perangkat lunak yang mendorong kolaborasi antara semua peserta proyek. BDD adalah evolusi dari test-driven development (TDD) dan acceptance test-driven planning.

Prinsip inti dari BDD mengatakan “bahwa orang bisnis dan teknologi harus merujuk ke sistem yang sama dengan cara yang sama”. Untuk mencapai tujuan ini, natural language diperlukan untuk menentukan perilaku sistem, agar sistem memungkinkan untuk: (a) pelanggan agar bisa menentukan requirement dari perspektif bisnis, (b) analis bisnis untuk melampirkan contoh konkret (skenario atau tes penerimaan) yang menjelaskan perilaku sistem, dan (c) pengembang untuk mengimplementasikan perilaku sistem yang diperlukan menggunakan cara TDD. 

Penggunaan behaviour specification pada pengembangan perangkat lunak akan membuat user dapat ikut berpartisipasi secara langsung dalam proses pengujian tanpa perlu memikirkan bagaimana cara melakukan pengujian sebuah perangkat lunak, karena akan langsung diarahkan oleh developer-nya cara untuk melakukan pengujian bagi pengguna.

Tahap pengujian merupakan tahap yang perlu dilakukan sebelum perangkat lunak disebarluaskan dan dengan adanya pengujian \textit{developer} akan meminimalisir segala kesalahan yang mungkin timbul pada perangkat lunak yang sedang dikembangkan. Dengan adanya pengujian berbasis \textit{behaviour specification} pengguna bisa ikut serta dalam pengembangan suatu perangkat lunak.

Pada skripsi ini akan dibangun sebuah perangkat pengujian yang mengimplementasi \textit{behaviour specification} untuk menguji sebuah perangkat lunak.

\section{Rumusan Masalah}
\label{sec:rumusan}
\begin{itemize}
\item Bagaimana cara kerja pengerjaan Pengujian Berbasis Behaviour Specification?
\item Bagaimana implementasi perangkat lunak yang dapat menguji program berbasis Behaviour Specification ?	
\end{itemize}

\section{Tujuan}
\label{sec:tujuan}
\begin{itemize}
\item Mempelajari cara kerja Pengujian Berbasis Behaviour Specification
\item Menghasilkan perangkat penguji yang dapat menguji sebuah perangkat lunak berbasiskan \textit{behaviour specification}.
\end{itemize}

\section{Batasan Masalah}
\label{sec:batasan}
Mengingat banyaknya perkembangan yang bisa ditemukan dalam permasalahan ini, maka perlu adanya batasan-batasan masalah yang jelas mengenai apa yang dibuat dan diselesaikan dalam program ini. Adapun batasan-batasan masalah pada penelitian ini sebagai berikut :

\section{Metodologi}
\label{sec:metlit}
Bagian-bagian pekerjaan skripsi ini adalah sebagai berikut :
\begin{enumerate}
\item Melakukan Studi literatur mengenai pengujian perangkat lunak, \textit{unit testing}, dan \textit{behaviour specification}.
\item Melakukan eksplorasi perangkat lunak.
\item Menganalisis kebutuhan perangkat lunak.
\item Membuat rancangan desain perangkat lunak, basis data, antarmuka perangkat lunak.
\item Implementasi perangkat lunak yang menerima \textit{input behaviour spesification}, mengolah, dan mengubah ke bentuk \textit{unit testing}.
\item Melakukan pengujian terhadap perangkat lunak.
\item Melaporkan hasil penelitian dalam bentuk dokumen skripsi.
\end{enumerate}

\section{Sistematika Pembahasan}
\label{sec:sispem}
\begin{itemize}

\item BAB I. PENDAHULUAN\\
Bab ini berisi tentang latar belakang masalah, rumusan
masalah, tujuan, batasan masalah, dan metodologi.
\item BAB II. LANDASAN TEORI\\
Bab 2 memuat uraian tentang teori yang akan digunakan pada pembuatan perangkat lunak.
\end{itemize}